\documentclass[journal,12pt,twocolumn]{IEEEtran}

\usepackage{setspace}
\usepackage{gensymb}
\singlespacing
\usepackage[cmex10]{amsmath}

\usepackage{amsthm}

\usepackage{mathrsfs}
\usepackage{txfonts}
\usepackage{stfloats}
\usepackage{bm}
\usepackage{cite}
\usepackage{cases}
\usepackage{subfig}

\usepackage{longtable}
\usepackage{multirow}

\usepackage{enumitem}
\usepackage{mathtools}
\usepackage{steinmetz}
\usepackage{tikz}
\usepackage{circuitikz}
\usepackage{verbatim}
\usepackage{tfrupee}
\usepackage[breaklinks=true]{hyperref}
\usepackage{graphicx}
\usepackage{tkz-euclide}

\usetikzlibrary{calc,math}
\usepackage{listings}
    \usepackage{color}                                            %%
    \usepackage{array}                                            %%
    \usepackage{longtable}                                        %%
    \usepackage{calc}                                             %%
    \usepackage{multirow}                                         %%
    \usepackage{hhline}                                           %%
    \usepackage{ifthen}                                           %%
    \usepackage{lscape}     
\usepackage{multicol}
\usepackage{chngcntr}

\DeclareMathOperator*{\Res}{Res}

\renewcommand\thesection{\arabic{section}}
\renewcommand\thesubsection{\thesection.\arabic{subsection}}
\renewcommand\thesubsubsection{\thesubsection.\arabic{subsubsection}}

\renewcommand\thesectiondis{\arabic{section}}
\renewcommand\thesubsectiondis{\thesectiondis.\arabic{subsection}}
\renewcommand\thesubsubsectiondis{\thesubsectiondis.\arabic{subsubsection}}


\hyphenation{op-tical net-works semi-conduc-tor}
\def\inputGnumericTable{}                                 %%

\lstset{
%language=C,
frame=single, 
breaklines=true,
columns=fullflexible
}
\begin{document}


\newtheorem{theorem}{Theorem}[section]
\newtheorem{problem}{Problem}
\newtheorem{proposition}{Proposition}[section]
\newtheorem{lemma}{Lemma}[section]
\newtheorem{corollary}[theorem]{Corollary}
\newtheorem{example}{Example}[section]
\newtheorem{definition}[problem]{Definition}

\newcommand{\BEQA}{\begin{eqnarray}}
\newcommand{\EEQA}{\end{eqnarray}}
\newcommand{\define}{\stackrel{\triangle}{=}}
\bibliographystyle{IEEEtran}
\raggedbottom
\setlength{\parindent}{0pt}
\providecommand{\mbf}{\mathbf}
\providecommand{\pr}[1]{\ensuremath{\Pr\left(#1\right)}}
\providecommand{\qfunc}[1]{\ensuremath{Q\left(#1\right)}}
\providecommand{\sbrak}[1]{\ensuremath{{}\left[#1\right]}}
\providecommand{\lsbrak}[1]{\ensuremath{{}\left[#1\right.}}
\providecommand{\rsbrak}[1]{\ensuremath{{}\left.#1\right]}}
\providecommand{\brak}[1]{\ensuremath{\left(#1\right)}}
\providecommand{\lbrak}[1]{\ensuremath{\left(#1\right.}}
\providecommand{\rbrak}[1]{\ensuremath{\left.#1\right)}}
\providecommand{\cbrak}[1]{\ensuremath{\left\{#1\right\}}}
\providecommand{\lcbrak}[1]{\ensuremath{\left\{#1\right.}}
\providecommand{\rcbrak}[1]{\ensuremath{\left.#1\right\}}}
\theoremstyle{remark}
\newtheorem{rem}{Remark}
\newcommand{\sgn}{\mathop{\mathrm{sgn}}}
\providecommand{\abs}[1]{\left\vert#1\right\vert}
\providecommand{\res}[1]{\Res\displaylimits_{#1}} 
\providecommand{\norm}[1]{\left\lVert#1\right\rVert}
%\providecommand{\norm}[1]{\lVert#1\rVert}
\providecommand{\mtx}[1]{\mathbf{#1}}
\providecommand{\mean}[1]{E\left[ #1 \right]}
\providecommand{\fourier}{\overset{\mathcal{F}}{ \rightleftharpoons}}
%\providecommand{\hilbert}{\overset{\mathcal{H}}{ \rightleftharpoons}}
\providecommand{\system}{\overset{\mathcal{H}}{ \longleftrightarrow}}
	%\newcommand{\solution}[2]{\textbf{Solution:}{#1}}
\newcommand{\solution}{\noindent \textbf{Solution: }}
\newcommand{\cosec}{\,\text{cosec}\,}
\providecommand{\dec}[2]{\ensuremath{\overset{#1}{\underset{#2}{\gtrless}}}}
\newcommand{\myvec}[1]{\ensuremath{\begin{pmatrix}#1\end{pmatrix}}}
\newcommand{\mydet}[1]{\ensuremath{\begin{vmatrix}#1\end{vmatrix}}}
\numberwithin{equation}{subsection}
\makeatletter
\@addtoreset{figure}{problem}
\makeatother
\let\StandardTheFigure\thefigure
\let\vec\mathbf
\renewcommand{\thefigure}{\theproblem}
\def\putbox#1#2#3{\makebox[0in][l]{\makebox[#1][l]{}\raisebox{\baselineskip}[0in][0in]{\raisebox{#2}[0in][0in]{#3}}}}
     \def\rightbox#1{\makebox[0in][r]{#1}}
     \def\centbox#1{\makebox[0in]{#1}}
     \def\topbox#1{\raisebox{-\baselineskip}[0in][0in]{#1}}
     \def\midbox#1{\raisebox{-0.5\baselineskip}[0in][0in]{#1}}
\vspace{3cm}
\title{CBSE Maths Questions 2007}
\author{Priyanka - EE21MTECH12002}
\maketitle
\newpage
\bigskip
\renewcommand{\thefigure}{\theenumi}
\renewcommand{\thetable}{\theenumi}
Download all python codes from 
\begin{lstlisting}
https://github.com/PeriPriyanka/cbsemathsquestions/2007/10/matrices/codes/solutions
\end{lstlisting}
%
Get latex-tikz codes from 
%
\begin{lstlisting}
https://github.com/PeriPriyanka/cbsemathsquestions/2007/10/matrices/solutions
\end{lstlisting}

\begin{enumerate}

\item (CBSE 2007-Question 2)
solve the values of x and y.
\begin{align}
&x+\displaystyle\frac{6}{y}=6 & \\ 
&3x-\displaystyle\frac{8}{y}=5&
\end{align}
\solution Consider the equations  given in the problem statement.
\begin{align}
&x+\displaystyle\frac{6}{y}=6 \label{eq:0.0.3} &\\
&3x-\displaystyle\frac{8}{y}=5 \label{eq:0.0.4} &
\end{align}
The solution can be found by solving the above system of linear equations.\\ 
System of linear equations are defined as 
\begin{align}
\vec{Ax=B}\label{eq:0.0.5}
\end{align}
From the equations \eqref{eq:0.0.3} and \eqref{eq:0.0.4}, 
\begin{align}
&\vec{A}= \myvec{1 &6\\3  & -8}&\\
\medskip
&\vec{x}= \myvec{ x\\ \displaystyle\frac{1}{y}}&\\
\medskip
&\vec{B}= \myvec{6\\5} & 
\end{align} 
Substituting the values of $\vec{A}$, $\vec{x}$ and $\vec{B}$ in the equation \eqref{eq:0.0.5}
We get,
\begin{align}
\myvec{1&6\\3&-8} \myvec{x\\\displaystyle\frac{1}{y}}= \myvec{6\\5}
\end{align}
Considering the augmented matrix 
 \begin{align}
  &\myvec{1&6&6\\3&-8&5}&\\ 
& \xleftrightarrow[]{ R_2 \leftarrow R_2 - 3R_1}
  \myvec{1&6&6\\0&-26&-13}&\\
  & \xleftrightarrow[]{ R_1 \leftarrow 13R_1 + 3R_2}
  \myvec{13&0&39\\0&-26&-13}&\\
   & \xleftrightarrow[]{ R_1 \leftarrow R_1/3,R_2 \leftarrow R_2/-26}
  \myvec{1&0&3\\0&1&0.5}&
 \end{align}
 \begin{align}
&\myvec{1&0\\0&1} \myvec{x\\\displaystyle\frac{1}{y}}= \myvec{0\\0.5}& \label{eq:0.0.14}
\medskip
\end{align}
From the above equation \eqref{eq:0.0.14} we get,
\begin{align}
&x=3&\\
&y=2&
\end{align}
Therefore, x=3 and y= 2 are solutions to the given equations.
\bigskip
\item (CBSE 2007-Question 3)
solve the values of x and y
\begin{align}
\displaystyle\frac{x+1}{2}+\displaystyle\frac{y-1}{3}=8\\
\displaystyle\frac{x-1}{3}+\displaystyle\frac{y+1}{2}=9\end{align}

\solution Consider the equations given in the problem statement.
\begin{align}
\displaystyle\frac{x+1}{2}+\displaystyle\frac{y-1}{3}=8\\
\displaystyle\frac{x-1}{3}+\displaystyle\frac{y+1}{2}=9
\end{align}
The above equations can be rearranged as the following equations
\begin{align}
3x+2y=47 \label{eq:0.0.21}\\
2x+3y=53 \label{eq:0.0.22}
\end{align}
The solution can be found by solving the above system of linear equations.\\ 
System of linear equations are defined as 
\begin{align}
\vec{Ax=B} \label{eq:0.0.23}
\end{align}
From the equations \eqref{eq:0.0.21} and \eqref{eq:0.0.22}, 
\begin{align}
&\vec{A}= \myvec{3 &2\\2 & 3}&\\
\medskip
&\vec{x}= \myvec{ x\\ y}&\\
\medskip
&\vec{B}= \myvec{47\\53}&  
\end{align} 
Substituting the values of $\vec{A}$, $\vec{x}$ and $\vec{B}$ in the equation \eqref{eq:0.0.23}
We get,
\begin{align}
&\myvec{3&2\\2&3} \myvec{x\\y}= \myvec{47\\53}&
\end{align}
Considering the augmented matrix 
 \begin{align}
& \myvec{3&2&47\\2&3&53}&
 \\
&\xleftrightarrow[]{ R_2 \leftarrow 3R_2 - 2R_1}
 \myvec{3&2&47\\0&5&65}&\\
 &\xleftrightarrow[]{ R_1 \leftarrow 5R_1 - 2R_2}
 \myvec{15&0&105\\0&5&65}&\\
 &\xleftrightarrow[]{ R_1 \leftarrow R_1/15,R_2\leftarrow R_2/5}
 \myvec{1&0&7\\0&1&13}&
 \end{align}
 \begin{align}
&\myvec{1&0\\0&1} \myvec{x\\y}= \myvec{7\\13}& \label{eq:0.0.32}
\medskip
\end{align}
By solving equation \eqref{eq:0.0.32} we get,
\begin{align}
&x=7&\\
&y= 13&
\end{align}
Therefore, x=7 and y= 13 are solutions to the given equations. 
\bigskip
\item (CBSE 2007-Question 21) Show that the points given below are vertices of an isosceles right angle triangle.
\begin{align}
\myvec{7\\10}, \myvec{-2\\5} \text{and} \myvec{3\\-4}
\end{align}

\solution Consider the given points as vectors,
\begin{align}
&\vec{A}= \myvec{7\\10}&\\
&\vec{B}=\myvec{-2\\5}&\\
&\vec{C}=\myvec{3\\-4}&
\end{align}
For a triangle to be an isosceles, any two sides of the triangle should be equal.
For finding a triangle to be isosceles and right angle, we consider,
\begin{align}
&\vec{A-B}= \myvec{7\\10} - \myvec{-2\\5} = \myvec{9\\5}&\\
&\vec{B-C}= \myvec{-2\\5} - \myvec{3\\-4} = \myvec{-5 \\ 9}&\\
&\vec{C-A}=  \myvec{3\\-4} - \myvec{7\\10} = \myvec{-4\\ -14}&
\end{align}
\begin{align}
&\vec{(A-B)}^T\vec{(B-C)} = \myvec{9&5}\myvec{-5\\9} &\\&= -45+45 = 0 \label{eq:0.0.43}&\\
& \vec{(C-A)}^T\vec{(A-B)} = \myvec{-4&-14}\myvec{9\\5}&\\& = -36-70 = -106 \label{eq:0.0.45}&\\
 &\vec{(B-C)}^T\vec{(C-A)} = \myvec{-5&9}\myvec{-4\\-14}&\\& = 20-126 = -106 \label{eq:0.0.47}&
\end{align}
From the equation \eqref{eq:0.0.43} \begin{align} \vec{A-B} \perp \vec{B-C}\end{align} Therefore $\measuredangle{B} = 90\degree$.
From the equations \eqref{eq:0.0.45} and \eqref{eq:0.0.47} $\measuredangle{CAB}=\measuredangle{BCA}$.
Therefore, $\triangle{ABC}$ is an isosceles right angle triangle with sides AB=BC and right angle at B.
\bigskip
\item (CBSE 2007-Question 22) In what ratio does the line \begin{align} \myvec{1&-1}\vec{x}=2 \label{eq:0.0.49}\end{align} divides the line segment joining \begin{align}\myvec{3 \\-1} \text{and} \myvec{8\\9} \label{eq:0.0.50}\end{align}

\solution Consider the line \begin{align} \vec{n}^T\vec{x}=c \end{align} divides the line segment $\vec{A}$ and $\vec{B} $ in $k:1$ ratio.
$\vec{p} $ is point of intersection of two lines.\\ 
From the section formula we can write,
\begin{align}
&\vec{p} = \displaystyle\frac{1}{k+1}\left[\vec{A}+ k\vec{B}\right]&\\
 \end{align}
The point $\vec{p}$ passes through the line $\vec{n}^T\vec{x}=c$, therefore,
\begin{align}
&\vec{n}^T\vec{p}=c&\\
&\vec{n}^T\left(\displaystyle\frac{\vec{A}+k\vec{B}}{k+1} \right)=c&
\end{align}
Solving for k, we get,
\begin{align}
&k=\displaystyle\frac{c-\vec{n}^T\vec{A}}{\vec{n}^T\vec{B}-c} \label{eq:0.0.56} &
\end{align}
From the equations \eqref{eq:0.0.49} and \eqref{eq:0.0.50},
\begin{align}
& \vec{n}^T = \myvec{1&-1}&\\
& \vec{A} = \myvec{3 \\-1}&\\
& \vec{B} = \myvec{8\\9}
\end{align}
Substituting the above values in the equation \eqref{eq:0.0.56}, we get,
\begin{align}
& k = \displaystyle\frac{2-\myvec{1&-1}\myvec{3\\-1}}{\myvec{1&-1}\myvec{8\\9} - 2}&\\
&k = \displaystyle\frac{2}{3}
\end{align}
Therefore, the line \begin{align} \myvec{1&-1}\vec{x}=2 \end{align} divides the line segment joining \begin{align}\myvec{3 \\-1} \text{and} \myvec{8\\9} \end{align} in 2:3 ratio.
\end{enumerate}
\end{document}