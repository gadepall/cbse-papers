\let\negmedspace\undefined
\let\negthickspace\undefined
\documentclass[journal,12pt,twocolumn]{IEEEtran}
\usepackage{gensymb}
\usepackage{amssymb}
\usepackage[cmex10]{amsmath}
\usepackage{amsthm}
\usepackage[export]{adjustbox}
\usepackage{bm}
\usepackage{longtable}
\usepackage{enumitem}
\usepackage{mathtools}
\usepackage[breaklinks=true]{hyperref}
\usepackage{listings}
\usepackage{color}                                            %%
\usepackage{array}                                            %%
\usepackage{longtable}                                        %%
\usepackage{calc}                                             %%
\usepackage{multirow}                                         %%
\usepackage{hhline}                                           %%
\usepackage{ifthen}                                           %%
\usepackage{lscape}     
\usepackage{multicol}
% \usepackage{enumerate}
\DeclareMathOperator*{\Res}{Res}
\renewcommand\thesection{\arabic{section}}
\renewcommand\thesubsection{\thesection.\arabic{subsection}}
\renewcommand\thesubsubsection{\thesubsection.\arabic{subsubsection}}
\renewcommand\thesectiondis{\arabic{section}}
\renewcommand\thesubsectiondis{\thesectiondis.\arabic{subsection}}
\renewcommand\thesubsubsectiondis{\thesubsectiondis.\arabic{subsubsection}}
\hyphenation{op-tical net-works semi-conduc-tor}
\def\inputGnumericTable{}                                 %%

\lstset{
frame=single, 
breaklines=true,
columns=fullflexible
}
\begin{document}
\newtheorem{theorem}{Theorem}[section]
\newtheorem{problem}{Problem}
\newtheorem{proposition}{Proposition}[section]
\newtheorem{lemma}{Lemma}[section]
\newtheorem{corollary}[theorem]{Corollary}
\newtheorem{example}{Example}[section]
\newtheorem{definition}[problem]{Definition}
\newcommand{\BEQA}{\begin{eqnarray}}
\newcommand{\EEQA}{\end{eqnarray}}
\newcommand{\define}{\stackrel{\triangle}{=}}
\bibliographystyle{IEEEtran}
\providecommand{\mbf}{\mathbf}
\providecommand{\pr}[1]{\ensuremath{\Pr\left(#1\right)}}
\providecommand{\qfunc}[1]{\ensuremath{Q\left(#1\right)}}
\providecommand{\sbrak}[1]{\ensuremath{{}\left[#1\right]}}
\providecommand{\lsbrak}[1]{\ensuremath{{}\left[#1\right.}}
\providecommand{\rsbrak}[1]{\ensuremath{{}\left.#1\right]}}
\providecommand{\brak}[1]{\ensuremath{\left(#1\right)}}
\providecommand{\lbrak}[1]{\ensuremath{\left(#1\right.}}
\providecommand{\rbrak}[1]{\ensuremath{\left.#1\right)}}
\providecommand{\cbrak}[1]{\ensuremath{\left\{#1\right\}}}
\providecommand{\lcbrak}[1]{\ensuremath{\left\{#1\right.}}
\providecommand{\rcbrak}[1]{\ensuremath{\left.#1\right\}}}
\theoremstyle{remark}
\newtheorem{rem}{Remark}
\newcommand{\sgn}{\mathop{\mathrm{sgn}}}
\providecommand{\abs}[1]{\left\vert#1\right\vert}
\providecommand{\res}[1]{\Res\displaylimits_{#1}} 
\providecommand{\norm}[1]{\left\lVert#1\right\rVert}
%\providecommand{\norm}[1]{\lVert#1\rVert}
\providecommand{\mtx}[1]{\mathbf{#1}}
\providecommand{\mean}[1]{E\left[ #1 \right]}
\providecommand{\fourier}{\overset{\mathcal{F}}{ \rightleftharpoons}}
%\providecommand{\hilbert}{\overset{\mathcal{H}}{ \rightleftharpoons}}
\providecommand{\system}{\overset{\mathcal{H}}{ \longleftrightarrow}}
	%\newcommand{\solution}[2]{\textbf{Solution:}{#1}}
\newcommand{\solution}{\noindent \textbf{Solution: }}
\newcommand{\cosec}{\,\text{cosec}\,}
\providecommand{\dec}[2]{\ensuremath{\overset{#1}{\underset{#2}{\gtrless}}}}
\newcommand{\myvec}[1]{\ensuremath{\begin{pmatrix}#1\end{pmatrix}}}
\newcommand{\mydet}[1]{\ensuremath{\begin{vmatrix}#1\end{vmatrix}}}
\numberwithin{equation}{subsection}
\makeatletter
\@addtoreset{figure}{problem}
\makeatother
\let\StandardTheFigure\thefigure
\let\vec\mathbf
\renewcommand{\thefigure}{\theproblem}
\def\putbox#1#2#3{\makebox[0in][l]{\makebox[#1][l]{}\raisebox{\baselineskip}[0in][0in]{\raisebox{#2}[0in][0in]{#3}}}}
     \def\rightbox#1{\makebox[0in][r]{#1}}
     \def\centbox#1{\makebox[0in]{#1}}
     \def\topbox#1{\raisebox{-\baselineskip}[0in][0in]{#1}}
     \def\midbox#1{\raisebox{-0.5\baselineskip}[0in][0in]{#1}}
\vspace{3cm}
\title{
	10th CBSE MATHEMATICS 2018
}
\author{ Keshav Roy
	\thanks{}
	
}
\maketitle
\newpage
\bigskip
\renewcommand{\thefigure}{\theenumi}
\renewcommand{\thetable}{\theenumi}
\section{Section A}
\renewcommand{\theequation}{\theenumi}
\begin{enumerate}[label=\thesection.\arabic*.,ref=\thesection.\theenumi]
\numberwithin{equation}{enumi}
\item Find the value of $k$ for which the roots of a quadratic equation $(k-5)x^2 + 2 (k-5)x + 2=0 $ are equal ?\\
\item Find the value of y for which the distance between the points (2, -3)and (10, y)is 10 units.\\

\item Write whether the rational number $\frac{13}{3125} $ has a decimal expansion which is terminating or non-terminating repeating.\\

\item Write the $n^th$ term of the A.P $\frac{1}{k},\frac{1+k}{k},\frac{1+2k}{k},...$\\

\item If $ sin \theta + cos\theta = \sqrt{2}cos (90\degree - \theta),$ find the value of cot$\theta$. \\

\item DE is drawn parallel to the base BC of $\triangle ABC$,meeting AB at D and AC at E if $\frac{AB}{CD}=4$and CE=2cm,find AE.\\
\end{enumerate}
\section{Section B}
\renewcommand{\theequation}{\theenumi}
\begin{enumerate}[label=\thesection.\arabic*.,ref=\thesection.\theenumi]
\numberwithin{equation}{enumi}
 \item A bag contains 5 red balls and some blue balls.If the probability of drawing a blue ball from the bag is three times that of the red ball, find the number of blue balls in the bag.\\

\item The $5^th$ and $15^th$ terms of an A.P are 13 and -17 respectively.Find the sum of first 21 terms of the A.P.\\

\item Using Euclid's Division Algorithm, find the HCF of 225 and 867\\

\item If the point $(0,2)$ is equidistant
t from the points$(3,k)$and $(k,5)$find the value of k.\\

\item Find the value of 'a' for which the pair of linear equation $2x+3y=7$ and $4x+ay=14$ has infinitely many solutions.\\

\item A card is drawn at random from a well shuffled pack of 52 paying cards.Find the probability of getting (i) a red king (ii) a queen or a jack.\\
\end{enumerate}
\section{Section C}
\renewcommand{\theequation}{\theenumi}
\begin{enumerate}[label=\thesection.\arabic*.,ref=\thesection.\theenumi]
\numberwithin{equation}{enumi}
\item Show that any positive odd integer is of the form 4q + 1 or 4q + 3 for some integer q.\\

\item The ten's digit of a number is twice its unit's digit. The number obtained by interchanging the digits is 36 less than the original number.Find the original number.\\

\item (i)The line segment joining the points $A(2,1)$and $B(5,-8)$ is trisected at the points P and Q, where P is nearer to A if P lies on the line $2x-y+k=0$,find the value of k.\\
\begin{align}
    \centering \vec{OR}\nonumber
\end{align}
\newline (ii) The x-coordinate of a points P is twice its y-coordinate.If P is equidistant from the point $Q(2,-5)$ and $R(-3,6),$find the coordinates.\\

\item Show that $1$,$\frac{1}{2},$ and $-2$ are the zeroes of the polynomial $2x^3+x^2-5x+2.$\\

\item Prove that the angle between the two tangents draws from an external points to a circle is supplementary to the angle subtended by the line-segment joining the points of contact at the center.\\

\item S and T are points on the sides PR and QR of $\triangle PQR $ Such that  $\angle P $ = $\angle RTS$.Show that $\triangle RPQ \sim \triangle RTS$.\\ 
\begin{align}
    \centering \vec{OR}\nonumber
\end{align}
In an equilateral $\triangle ABC$, D is a point on the side BC such that BD =$\frac{1}{3}BC$, Prove that $9AD^2 =7AB^2.$\\

\item Prove that : \\ $\frac{1}{\cosec\theta + \cot\theta}-\frac{1}{\sin\theta}=\frac{1}{\sin\theta}-\frac{1}{\cosec\theta-\cot\theta}$
\begin{align}
    \centering \vec{OR}\nonumber
\end{align}
If $\tan\theta + \sin\theta = m$,$\tan\theta-\sin\theta=n$ show that m$^2$-$n^2=4\sqrt{mn}$
\item A chord of a circle, of radius 15 cm,subtends an angel of $60\degree$ at the centre of the circle. Find the area of major and minor segments (Take $\pi=3.14, \sqrt{3}=1.73$)
\item A sphere of diameter 12 cm is dropped in a right circular cylindrical vessel, partly filled with water, If the sphere is completely submerged in water, the water level in the vessel. rises by $3\frac{5}{9}$cm. Find the diameter of the cylindrical vessel.
\begin{align}
    \centering \vec{OR}\nonumber
\end{align}
A cylinder whose height is two-third of its diameter, has the same volume as that of a sphere of radius 4 cm. Find the radius of base of the cylinder.
\item The following table gives the daily income of 50 labourers :\\
\begin{table}[htb]
	\centering
	\resizebox{\columnwidth}{!}{
\begin{tabular}{ |c|c|c|c|c|c|c|c| } 
\hline
{Class :}& 0 – 10 &10 – 20 & 20 – 30&30 – 40 &40 – 50&50 – 60&60-70\\
\hline
{Frequency :}& 5 & 15 & 20& 23&17&11&9\\ 
\hline
\end{tabular}
}
Find the mean and mode of the above data.\\
\end{table}
\item Two taps together can fill a tank in 6 hours. The tap of larger diameter takes 9 hours  less than the smaller one to fill the tank separately. Find the time in which each tap can fill the tank separately.
\begin{align}
    \centering \vec{OR}\nonumber
\end{align}
\item Solve for $x : \frac{x+1}{x-1}-\frac{x-1}{x+1}=\frac{5}{6}$,$x\neq1,-1$\\
\item Prove that the ratio of the areas of two similar triangles is equal to the square of the ratio of their corresponding sides.
\begin{align}
    \centering \vec{OR}\nonumber
\end{align}
Prove that in a triangle, if the square of one side is equal to sum of the square of the other two sides, the angle opposite the first side is a right angle.\\
\item Write the steps of construction for drawing a $\triangle$ABC in which BC = 8 cm, $\angle$B=45$\degree$ and $\angle$C=30$\degree$. Now write the steps of construction for drawing a triangle whose sides are $\frac{3}{4}$ of the corresponding sides of $\triangle$ABC.\\
\item The sum of the first n terms of an A.P. is $5n^2 + 3n$. If its m$^{th}$ term is 168, find the value of m. Also find the 20$^{th}$ term of the A.P.
\begin{align}
    \centering \vec{OR}\nonumber
\end{align}
The 4$^{th}$ and the last terms of an A.P. are 11 and 89 respectively. If there are 30 terms in the A.P., find the A.P. and its 23$^{rd}$ term.\\
\item Prove that : \\ $(\frac{\sin A}{1-\cos A}-\frac{1-\cos A}{\sin A})$ . $(\frac{\cos A}{1-\sin A}-\frac{1-\sin A}{\cos A}) = 4$ \\
\item A statue, 1.46 m tall, stands on a pedestal. From a point on the ground the angle of elevation of the top of the stature is 60$\degree$ and from the same point angle of elevation of the top of the pedestal is 45$\degree$. Find the height of the pedestal. (use $\sqrt{3=1.73}$ ) \\
\item Sudhakar donated 3 cylindrical drums to store cereals to an orphanage. If radius of each drum is 0.7 m and height 2 m, find the volume of each drum. If m$^3$, find the amount spent by Sudhakar for orphanage. What value is exhibited in the question. (Use $\pi = \frac{22}{7}$ ).\\

\item The median of the following data is 52.5. If the total frequency is 100, find the values of x and y.\\
\begin{table}[htb]
	\centering
	\resizebox{\columnwidth}{!}{
\begin{tabular}{ |c|c|c|c|c|c|c|c|c|c|c|c| } 
\hline
{Class :}&0 – 10 &10 – 20 &20 – 30 &30 – 40 &40 – 50 &50 – 60 &60-70 &70-80 &80 -90 &90 -100\\
\hline
{Frequency :}& 2 & 5 & x & 12 & 17 & 20 & y & 7 & 9 & 4\\ 
\hline
\end{tabular}
}
\end{table}
\end{enumerate}
\end{document}
